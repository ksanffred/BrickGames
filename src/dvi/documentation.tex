\documentclass{article}
\usepackage{enumitem}
\usepackage{hyperref}

\begin{document}

\title{Documentation for Snake Game v1.0}
\date{}
\maketitle

\section{Introduction}

The Snake Game v1.0 project is an implementation of the classic arcade game Snake. The game logic is implemented in C++20, and the project is built on the Model-View-Controller (MVC) architectural pattern. The user interface is available in two forms: a terminal interface and a graphical desktop interface.

\section{Project Structure}

\subsection{The Snake library (\texttt{src/snake\_game})}

\begin{itemize}[label=--]
    \item Source code files that implement the logic of the game Snake.
    \item This part represents the Model in the MVC pattern.
    \item Basic functions for interacting with the game board, controlling the snake, and handling user input.
    \item Implementation of a finite state machine to formalize the game's logic.
\end{itemize}

\subsection{Terminal interface (\texttt{src/gui/cli})}

\begin{itemize}[label=--]
    \item Source code files responsible for visualizing the game in the terminal.
    \item This part represents the View and Controller for the terminal version.
    \item Implementation of game field rendering, user input control, and current game state display.
\end{itemize}

\subsection{Graphical desktop interface (\texttt{src/gui/desktop})}

\begin{itemize}[label=--]
    \item Source code files for the graphical desktop interface.
    \item This part is built using the Qt framework, specifically QML for the user interface.
    \item It represents the View and Controller for the graphical version.
\end{itemize}

\section{Project Assembly}

The project uses a \texttt{make} build system with a Makefile including the following targets:

\begin{itemize}
    \item \texttt{all}: To build the project.
    \item \texttt{install}: Installing the program on the system.
    \item \texttt{uninstall}: Uninstalling a program from the system.
    \item \texttt{clean}: Clearing temporary files and folders.
    \item \texttt{dvi}: Creating a DVI file.
    \item \texttt{dist}: Create an archive containing the necessary files to build and use the program.
\end{itemize}

\section{Runtime Environment Requirements}

The project assumes the use of the C++20 programming language, a compatible compiler (like GCC or Clang), and the ncurses and Qt libraries for the terminal and graphical interfaces, respectively.

\section{Instructions for installation and run}

\begin{enumerate}
    \item \textbf{Installing dependencies:}
        \begin{itemize}
            \item Make sure you have a C++20-compatible compiler installed.
            \item Install the ncurses and Qt libraries.
        \end{itemize}
    \item \textbf{Project Build:}
        \begin{itemize}
            \item Run \texttt{make all} to build the project.
        \end{itemize}
    \item \textbf{Installation:}
        \begin{itemize}
            \item Run \texttt{make install} to install the program on the system.
        \end{itemize}
    \item \textbf{Run:}
        \begin{itemize}
            \item Execute \texttt{make run} to run the program.
        \end{itemize}
\end{enumerate}

\section{Using the program}

\begin{enumerate}
    \item \textbf{Management:}
        \begin{itemize}
            \item Use the ARROW KEYS to change the direction of the snake.
            \item Press P to pause the game.
            \item Press P again to unpause the game.
            \item Use ESC to exit the game.
        \end{itemize}
    \item \textbf{Game mechanics:}
        \begin{itemize}
            \item The snake moves continuously on the field.
            \item The snake's length increases when it eats an apple.
            \item A new apple appears randomly on the field after the previous one has been eaten.
        \end{itemize}
    \item \textbf{Ending the game:}
        \begin{itemize}
            \item The game ends when the snake touches the walls of the field or its own body.
            \item The player wins when they score 200 points.
        \end{itemize}
\end{enumerate}

\section{Testing}

The project includes unit tests. The test coverage of the library is at least 80\%.

\end{document}